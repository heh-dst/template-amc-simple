\documentclass[11pt,french,a4paper]{article}

\usepackage{enumitem}
\usepackage{graphicx}
\usepackage{listings}
\usepackage{multicol}
\usepackage{titling}
\usepackage{hyphenat}
\usepackage{booktabs}

\usepackage[francais,box,completemulti]{automultiplechoice}
\usepackage{hyperref}
\title{Examen}
\date{\today}
\author{François ROLAND}
\AMCrandomseed{1}
\hypersetup{
  pdftitle={\thetitle},
  pdfauthor={\theauthor},
  pdflang={fr-BE},
  hidelinks}
\usepackage{polyglossia}
\setdefaultlanguage{french}
\usepackage{csquotes}
\geometry{hmargin=2cm,headheight=2cm,headsep=.3cm,footskip=1cm,top=3cm,bottom=2cm}
\begin{document}

\scoringDefaultM{mz=1}

%%% preparation of the groups

% chktex-file 19
\setdefaultgroupmode{withoutreplacement}

%%% questions (available snippets: mcsa[h|mc], mcma[h|mc], gra)

%%% copies

\onecopy{1}{
  \setcounter{figure}{0}
  \setlength{\parindent}{0pt}
  %%% beginning of the header
  \noindent{{\LARGE{\nohyphens{\thetitle}}}}

  \vspace{2em}

  \noindent{\setlength{\parindent}{0pt}\hspace*{\fill}\AMCcodeGridInt{ID}{5}\hspace*{\fill}
  \begin{minipage}[b]{.6\linewidth}
    \(\longleftarrow{}\)\hspace{0pt plus 1cm} Codez votre matricule d'étudiant ci-contre et écrivez la date ainsi que votre nom et votre prénom tels qu'ils apparaissent sur votre document d'identité à l'intérieur des cadres ci-dessous.

    \vspace{1.5em}

    \fbox{\begin{minipage}[b]{\linewidth}%
      {\footnotesize Date \par}%
      \vspace{5mm}\dotfill%
      \vspace*{1mm}%
      \end{minipage}}

    \vspace{.5em}

    \namefield{\fbox{\begin{minipage}[b]{\linewidth}%
      {\footnotesize Nom et prénom \par}%
      \vspace{5mm}\namefielddots{}%
      \vspace*{1mm}%
      \end{minipage}}}
  \end{minipage}}

  \noindent\hrulefill{}
  \begin{itemize}[nosep]
    \item Répondez aux questions à l'aide d'une croix dans la case correspondant à votre réponse.
    \item Les questions avec le symbole~\multiSymbole{} peuvent avoir plusieurs réponses valides.
          Cochez toutes les bonnes réponses ou \textit{Aucune de ces réponses n'est correcte} pour avoir les points.
          Si une mauvaise réponse est cochée ou qu'il manque une réponse, la question ne rapporte aucun point.
    \item Il n'y a pas de points négatifs pour les mauvaises réponses.
    \item Une case complètement noircie est considérée comme une case non cochée.
          Utilisez ceci pour corriger une éventuelle erreur.
    \item Ce qui est écrit ou dessiné en dehors des cadres de réponse n'est pas pris en compte pour l'évaluation.
    \item Les cases sur fond gris sont réservées au correcteur, ne les cochez pas.
    \item Des pages vides ont été ajoutées à votre formulaire en guise de brouillon.
          Vous devez remettre le formulaire complet, feuilles de brouillon comprises, à la fin de votre évaluation.
    \item Afin de faciliter la lecture optique, utilisez un bic bleu foncé ou noir.
  \end{itemize}
  \noindent\hrulefill{}
  \vspace*{1.5em}

  %%% end of the header

  \cleargroup{all}

  %%% copy question groups to the all group for randomization (available snippet: cgr)

  \insertgroup{all}

  \AMCaddpagesto{4}
}


\end{document}
