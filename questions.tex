%%AMC:latex_engine=xelatex --shell-escape

\documentclass[11pt,french,a4paper,twoside]{article}

\usepackage{array,enumitem,graphicx,listings,multicol,multirow,titling,hyphenat,booktabs}

\usepackage[lang=FR,box,noshuffle,noshufflegroups,automarks,separateanswersheet]{automultiplechoice}
\usepackage{hyperref}
\title{Nom du cours}
\newcommand{\thesubtitle}{Interro sommative du \thedate{}}
\date{07/10/2024}
\author{François ROLAND}
\hypersetup{
  pdftitle={\thetitle},
  pdfauthor={\theauthor},
  pdflang={fr-BE},
  hidelinks}
\usepackage{fontspec}
\usepackage{libertine}
\usepackage{polyglossia}
\setdefaultlanguage{french}
\usepackage{csquotes}

\AMCrandomseed{1}
\AMCsetFoot{\thepage{}}

\setlength{\parindent}{0pt}

\begin{document}

%%% preparation of the groups

% chktex-file 19
\setdefaultgroupmode{withoutreplacement}

%%% questions (available snippets: mcsa[h|mc], gra, igs)
\element{q01}{
	\begin{question}{q01.1}
		Quelle est la couleur du cheval blanc d'Henri IV~?
		\begin{multicols}{2}
			\begin{choices}
				\correctchoice{blanc}
				\wrongchoice{noir}
				\wrongchoice{jaune}
				\wrongchoice{rouge}
				\lastchoices{}\columnbreak{}
				\wrongchoice{\textsc{aucune}}
				\wrongchoice{\textsc{toutes}}
				\wrongchoice{\textsc{manque}}
				\wrongchoice{\textsc{absurdité}}
			\end{choices}
		\end{multicols}
	\end{question}
}

%%% copies

\begin{examcopy}[2]
	\setcounter{figure}{0}
	%%% beginning of the header
	{\bf \thetitle{} \hfill{} \thesubtitle{}}

	\vspace{2ex}

	Durée~: 60~minutes.

	Aucun document n'est autorisé.
	Avoir un objet électronique (smartphone, montre, écouteurs\ldots{}) à portée de main pendant l'examen sera considéré comme une tentative de fraude.

	Commencez toujours par indiquer lisiblement votre nom et votre prénom sur la feuille de réponse.

	À Chaque question correspond \textbf{une} (et \textbf{une seule}) solution correcte.

	Chaque réponse correcte rapporte 1~point. Chaque réponse incorrecte, absente ou mal codée, 0~point.

	Vous pouvez inscrire ce que vous voulez sur le questionnaire et les feuilles de brouillon, car seule la feuille de réponse sera corrigée.
	Vous devez rendre toutes les feuilles reçues, y compris celles de brouillon, à la fin de votre évaluation.

	\hrulefill{}

	\vspace{2ex}

	%%% end of the header

	\cleargroup{all}

	%%% copy question groups to the all group for randomization (available snippet: cgr)
	\copygroup[1]{q01}{all}

	\insertgroup{all}

	\clearpage{}

	\begin{center}
		{\large{Feuille de brouillon}}
	\end{center}

	\AMCcleardoublepage{}

	\AMCformBegin{}
	\begin{center}\Large Feuille de réponse\end{center}

	{\bf \thetitle{}\hspace*{\fill}\thesubtitle{}}

	\vspace{1em}

	\begin{tabular}{@{}p{.6\textwidth}@{\hspace{.05\textwidth}}p{.35\textwidth}@{}}
		Nom et prénom                                                              & \hfill{}Matricule (si vous le connaissez) \\

		\fbox{\begin{minipage}{\linewidth}%
				      \vspace{1.5\baselineskip}\namefielddots{}%
				      \vspace*{1mm}%
			      \end{minipage}} & \hfill{}\AMCcodeGridInt[top]{Matricule}{6}\vspace{1ex}
	\end{tabular}

	\vspace{1em}

	Utilisez uniquement un \textbf{bic bleu ou noir} standard, pas d'autre couleur, ni de crayon, ni de feutre.

	\textbf{Noircissez entièrement une seule case par question}.

	Pour modifier votre réponse, utilisez du correcteur (liquide ou en ruban).
	Ne retracez jamais ni le contour d'une case ni sa lettre.
	N'entourez pas de case.

	\hrulefill{}

	\vspace{1em}
	%%% end of the answer sheet header

	\AMCform{}
	\AMCcleardoublepage{}

\end{examcopy}


\end{document}
