\documentclass[12pt,french,a4paper,oneside]{article}

\usepackage{amsmath,booktabs,graphicx,multicol,titling}

\usepackage[lang=FR,box,noshuffle,noshufflegroups,automarks,separateanswersheet]{automultiplechoice}

\usepackage{hyperref}

\title{Nom du cours}
\newcommand{\thesubtitle}{Interro sommative du \thedate{}}
\date{07/10/2024}
\author{François ROLAND}
\hypersetup{
  pdftitle={\thetitle},
  pdfauthor={\theauthor},
  pdflang={fr-BE},
  hidelinks}

  \usepackage{fontspec}
\usepackage[sfdefault]{roboto}
\usepackage{roboto-mono}

\usepackage{setspace}
\onehalfspacing{}

\usepackage[locale=FR]{siunitx}

\usepackage{polyglossia}
\setdefaultlanguage{french}

\usepackage{csquotes}

\AMCrandomseed{1}
\AMCsetFoot{\thepage{}}

\setlength{\parindent}{0pt}

\begin{document}

%%% preparation of the groups

% chktex-file 19
\setdefaultgroupmode{withoutreplacement}

%%% questions (available snippets: mcsa[h|mc], gra, igs)
\element{q01}{
	\begin{question}{q01.1}
		Quelle est la couleur du cheval blanc d'Henri IV~?
		\begin{multicols}{2}
			\begin{choices}
				\correctchoice{blanc}
				\wrongchoice{noir}
				\wrongchoice{jaune}
				\wrongchoice{rouge}
				\lastchoices{}\columnbreak{}
				\wrongchoice{aucune}
				\wrongchoice{toutes}
				\wrongchoice{manque}
				\wrongchoice{absurdité}
			\end{choices}
		\end{multicols}
	\end{question}
}

%%% copies

\begin{examcopy}[2]
	\setcounter{figure}{0}
	%%% beginning of the header
	{\bf \thetitle{} \hfill{} \thesubtitle{}}

	Durée~: 60~minutes.

	\vspace{1em}

	\begin{center}
		\fbox{Ne tournez pas cette page avant d'en avoir reçu l'autorisation.}
	\end{center}

	\vspace{1em}

	\textbf{\Large{}Consignes}

	\begin{itemize}
		\item Indiquer lisiblement votre nom et votre prénom sur la \textbf{feuille de réponse} qui se trouve à la fin du formulaire.
		\item À chaque question correspond \textbf{une} (et \textbf{une seule}) réponse correcte.
		\item Chaque réponse correcte vaut 1~point. Il n'y a pas de point négatif.
		\item Seule la feuille de réponse sera corrigée.
		      Vous pouvez donc écrire au recto et au verso des feuilles de question et sur les feuilles de brouillon.
		\item Sur la \textbf{feuille de réponse}, remplissez complètement la case correspondant à votre réponse.
		\item Utilisez uniquement un \textbf{bic bleu} ou \textbf{noir} standard non-effaçable pour faciliter l'archivage et la lecture optique.
		\item Pour modifier votre réponse, utilisez du correcteur (liquide ou en ruban).
		\item Les éventuelles zones grisées sont réservées à la correction.
	\end{itemize}

	\vfill{}

	\begin{center}
		\fbox{Ne tournez pas cette page avant d'en avoir reçu l'autorisation.}
	\end{center}

	\clearpage{}

	%%% end of the header

	\textbf{\Large{}Questions}

	\vspace{1em}

	\cleargroup{all}

	%%% copy question groups to the all group for randomization (available snippet: cgr)
	\copygroup[1]{q01}{all}

	\insertgroup{all}

	\clearpage{}

	\begin{center}
		{\large{Feuille de brouillon}}
	\end{center}

	\clearpage{}

	\AMCformBegin{}
	\begin{center}\Large Feuille de réponse\end{center}

	{\bf \thetitle{}\hspace*{\fill}\thesubtitle{}}

	\vspace{1em}

	Nom et prénom \\
	\namefield{\fbox{\begin{minipage}{\linewidth}%
				\vspace{1cm}\namefielddots{}%
				\vspace*{1mm}%
			\end{minipage}}}

	\vspace{1em}

	%%% end of the answer sheet header

	\AMCform{}
	\clearpage{}

\end{examcopy}


\end{document}
