\documentclass[11pt,french,a4paper,twoside]{scrartcl}

\usepackage{array}
\usepackage{enumitem}
\usepackage{graphicx}
\usepackage{listings}
\usepackage{multicol}
\usepackage{multirow}
\usepackage{titling}
\usepackage{hyphenat}
\usepackage{booktabs}

\usepackage[francais,box]{automultiplechoice}
\usepackage{hyperref}
\title{Examen}
\date{\today}
\author{François ROLAND}
\AMCrandomseed{1}
\AMCsetFoot{\thepage{}}
\hypersetup{
  pdftitle={\thetitle},
  pdfauthor={\theauthor},
  pdflang={fr-BE},
  hidelinks}
\usepackage{polyglossia}
\setdefaultlanguage{french}
\usepackage{csquotes}
\geometry{hmargin=2cm,headheight=2cm,headsep=.3cm,footskip=1cm,top=3cm,bottom=2cm}
\begin{document}

\scoringDefaultM{MAX=20,e=0,v=0,formula= (NBC==1?SCORECORRECT:SCOREWRONG),default.SCORECORRECT=20,default.SCOREWRONG=-20}

%%% preparation of the groups

% chktex-file 19
\setdefaultgroupmode{withoutreplacement}

%%% questions (available snippets: mcsa[h|mc], mcma[h|mc], gra)

%%% copies

\onecopy{1}{
  \setcounter{figure}{0}
  \setlength{\parindent}{0pt}
  %%% beginning of the header
  \noindent{%
    \fbox{\begin{minipage}[b]{.33\linewidth}%
        {\footnotesize Date \par}%
        \vspace{\baselineskip}\dotfill%
        \vspace*{1mm}%
      \end{minipage}}%
    \hspace{\fill}%
    \namefield{\fbox{\begin{minipage}[b]{.63\linewidth}%
          {\footnotesize Nom et prénom \par}%
          \vspace{\baselineskip}\namefielddots{}%
          \vspace*{1mm}%
        \end{minipage}}}}

  \vspace{\baselineskip}

  \begin{center}
    {\usekomafont{title}\huge{\nohyphens{\thetitle}}}
  \end{center}

  \AMCsection*{Consignes}
  \begin{itemize}
    \item Répondez aux questions en dessinant une croix avec un bic bleu foncé ou noir dans la case correspondant à votre choix.
    \item Vous devez accompagner chaque réponse d'un des degrés de certitude suivants.
          Vos points seront attribués en fonction de votre choix de degré de certitude selon le tableau ci-dessous.
          \begin{center}
            \begin{tabular}{llcccccc} \toprule
                                           &            & \multicolumn{6}{c}{Degré de certitude (en \%)}                                                              \\ \cmidrule{3-8}
                                           &            & \(0-25\)                                       & \(25-50\) & \(50-70\) & \(70-85\) & \(85-95\) & \(95-100\) \\ \midrule
              \multirow{2}{4.5em}{Réponse} & Correcte   & \(+13\)                                        & \(+16\)   & \(+17\)   & \(+18\)   & \(+19\)   & \(+20\)    \\ \cmidrule{2-8}
                                           & Incorrecte & \(+4\)                                         & \(+3\)    & \(+2\)    & \(0\)     & \(-6\)    & \(-20\)    \\ \bottomrule
            \end{tabular}
          \end{center}
          Le barème des points a été calculé de manière à ce que~:
          \begin{itemize}
            \item Dire la vérité soit la stratégie qui rapporte le plus de points;
            \item Ceux qui s'auto-évaluent bien gagnent plus de points que si l'on appliquait un barème
                  correctif tenant compte des probabilités d'avoir la réponse correcte en la devinant.\footnote{En général,
                    la majorité des étudiant.e.s s'autoestiment bien (c'est-a-dire avec réalisme) et sont avantagé.e.s
                    par les degrés de certitude~: leur score est meilleur que s'il avait été calculé sur la seule base
                    du nombre de réponses correctes.}
          \end{itemize}
    \item \textbf{Une case complètement noircie est considérée comme une case non cochée}.
    \item Ce qui est écrit ou dessiné en dehors des cadres de réponse n'est pas pris en compte pour l'évaluation.
    \item Les cases sur fond gris sont réservées au correcteur, ne les cochez pas.
    \item Les pages vides à la fin de votre formulaire sont des pages de brouillon.
          \textbf{Vous devez remettre le formulaire complet}, feuilles de brouillon comprises, à la fin de votre évaluation.
  \end{itemize}
  \clearpage

  \AMCsection*{Questions}
  %%% end of the header

  \cleargroup{all}

  %%% copy question groups to the all group for randomization (available snippet: cgr)

  \insertgroup{all}

  \AMCaddpagesto{4}
}


\end{document}
