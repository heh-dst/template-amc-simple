\documentclass[11pt,french,a4paper]{article}

\usepackage{enumitem}
\usepackage{graphicx}
\usepackage{listings}
\usepackage{multicol}
\usepackage{titling}
\usepackage{hyphenat}

\usepackage[francais,box,completemulti]{automultiplechoice}
\usepackage{hyperref}
\title{Projet d'intégration de développement}
\date{\today}
\author{François ROLAND}
\AMCrandomseed{1}
\hypersetup{
  pdftitle={\thetitle},
  pdfauthor={\theauthor},
  pdflang={fr-BE},
  hidelinks}
\usepackage{polyglossia}
\setdefaultlanguage{french}
\usepackage{csquotes}
\geometry{hmargin=2cm,headheight=2cm,headsep=.3cm,footskip=1cm,top=3cm,bottom=2cm}
\begin{document}

%%% preparation of the groups

% chktex-file 19
\setdefaultgroupmode{withoutreplacement}

%%% questions (available snippets: mcsa[h|mc], mcma[h|mc], gra)

%%% copies

\onecopy{16}{
  \setlength{\parindent}{0pt}
  %%% beginning of the header
  \noindent{{\LARGE{\nohyphens{\thetitle}}}}

  \vspace{2em}

  \noindent{\setlength{\parindent}{0pt}\hspace*{\fill}\AMCcodeGridInt{ID}{5}\hspace*{\fill}
  \begin{minipage}[b]{.6\linewidth}
    \(\longleftarrow{}\)\hspace{0pt plus 1cm} Codez votre matricule d'étudiant ci-contre et écrivez la date ainsi que vos nom et prénom tels qu'ils apparaissent sur le document d'identité présenté lors de votre inscription à l'intérieur des cadres ci-dessous.

    \vspace{1.5em}

    \fbox{\begin{minipage}[b]{\linewidth}%
      {\footnotesize Date \par}%
      \vspace{5mm}\dotfill%
      \vspace*{1mm}%
      \end{minipage}}

    \vspace{.5em}

    \namefield{\fbox{\begin{minipage}[b]{\linewidth}%
      {\footnotesize Nom et prénom \par}%
      \vspace{5mm}\namefielddots{}%
      \vspace*{1mm}%
      \end{minipage}}}
  \end{minipage}}

  \noindent\hrulefill{}
  \begin{itemize}[nosep]
    \item Répondez aux questions à l'aide d'une croix dans la case correspondant à votre réponse.
    \item Les questions avec le symbole \multiSymbole{} peuvent avoir zéro, une ou plusieurs bonnes réponses.
          Les autres ont une seule bonne réponse.
    \item Vous pouvez \enquote{décocher} une case en la noircissant, sans laisser d'espace blanc.
    \item Rien de ce qui est écrit ou dessiné en dehors des cases à cocher ne sera pris en compte.
    \item Afin de faciliter la lecture optique, utilisez un bic bleu ou noir.
  \end{itemize}
  \noindent\hrulefill{}
  \vspace*{1.5em}

  %%% end of the header

  \cleargroup{all}

  %%% copy question groups to the all group for randomization (available snippet: cgr)

  \insertgroup{all}

  \AMCcleardoublepage{}
}


\end{document}
